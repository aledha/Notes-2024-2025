In this section, we will derive a simplified electro-mechanical cardiac
model that is significantly numerically cheaper to compute than the full
three-dimensional model \eqref{eq:}.

We use the Ten Tusscher \cite{} cell model, the monodomain equations,
and the Land model \cite{}, to describe the electrophysiological (EP)
activity in a domain \(\Omega\), as well as the coupling to the
mechanical model. These can be summarized as \[
\begin{align}
\nabla \cdot(\mathbf{M}\nabla v) & =\chi C_{m}\frac{ \partial v }{ \partial t } +\chi I_\text{ion}(v,s), & \mathbf{x}\in \Omega , \\
\frac{ \partial s }{ \partial t } & =f(s,v,\lambda,\dot{\lambda},t), & \mathbf{x}\in \Omega, \\
\mathbf{n}\cdot(\mathbf{M}\nabla v) & =0, &\mathbf{x}\in \partial\Omega ,
\end{align}
\] where \(s\) is a collection of cell states and \(v\) is the
transmembrane potential. The Land model is included in system of
ordinary differential equations \eqref{eq:}, such that \(T_{a}\) can be
computed from a selection of the cell states \(s\). \eqref{eq:} and
\eqref{eq:} must, in general, be solved at every point in the domain.

With the assumption of a uniform electrical activation \(I_\text{stim}\)
and constant \(\lambda\) for each time \(t\), we want to show that if
\(v\) and \(s\) are spatially constant at \(t=0\), they will remain
spatially constant for every time \(t\).

Assume that \(v\) and \(s\) are constant at \(t=0\). Then it is easy to
see that \(\frac{ \partial s }{ \partial t }\) is constant at \(t=0\).
We also get that \(\nabla v=\boldsymbol{0}\) at \(t=0\), which reduces
the monodomain equation \eqref{eq:} to \[
\begin{equation}
\frac{ \partial v }{ \partial t } =-\frac{1}{C_{m}}I_\text{ion}(v,s)\quad \text{for } t=0,
\end{equation}
\] which also is constant across \(\Omega\). Since
\(\frac{ \partial s }{ \partial t }\) and
\(\frac{ \partial v }{ \partial t }\) are spatially constant at \(t=0\),
they will remain constant at all times \(t\). Further, \eqref{eq:} holds
for all times \(t\), and since \(\nabla v=\boldsymbol{0}\) for all
\(t\), the boundary condition \eqref{eq:} is always satisfied.

The numerical benefits from these assumptions are major. The nonlinear
partial differential equation \eqref{eq:} is reduced to one ordinary
differential equation \eqref{eq:}, and the system of ordinary
differential equations \eqref{eq:} needs only to be solved at one point
rather than every point/node in the domain. The fact that we can
describe the state of the system with only one set of cell states \(s\),
one value of the transmembrane potential \(v\), and one value for the
stretch ratio \(\lambda\), is why we refer to this type of model as a
\emph{zero-dimensional model}. These assumptions will be useful in
analyzing stability properties, from which we can infer on the stability
properties of the full three-dimensional model.

Since \(v\) is now controlled by a single ODE, we can include it in the
collection of cell states \(s\), such that the zero-dimensional model
simply reads \[
\begin{equation}
\frac{ \partial s }{ \partial t } =f(s,\lambda,\dot{\lambda},t).
\end{equation}
\]

\begin{center}\rule{0.5\linewidth}{0.5pt}\end{center}

Using the simplified model described above, we can solve for an active
tension \(T_{a}\). Now, we wish to derive a scheme to solve for the
stretch ratio \(\lambda\).

Let \(\Omega\) be a cuboidal slab, with fiber direction
\(\mathbf{f}_{0}=\begin{bmatrix}1&0&0\end{bmatrix}^T\). We assume that
the slab is transversely isotropic and incompressible, which leads to a
diagonal deformation matrix of the form \[
\begin{equation}
\mathbf{F}=\begin{bmatrix}
\lambda & 0 & 0 \\
0 & \frac{1}{\sqrt{ \lambda }} & 0 \\
0 & 0 & \frac{1}{\sqrt{ \lambda }} 
\end{bmatrix},
\end{equation}
\] and \(J=\det \mathbf{F}=1\). The right Cauchy-Green deformation
tensor becomes \[
\begin{equation}
\mathbf{C}=\begin{bmatrix}
\lambda^2 & 0 & 0 \\
0 & \frac{1}{\lambda} & 0 \\
0 & 0 & \frac{1}{\lambda} 
\end{bmatrix}.
\end{equation}
\] Then, the invariants simplify to \[
\begin{align}
I_{1} & =\mathrm{Tr}\mathbf{C}=\lambda^2+2\lambda^{-1}, \\
I_{4\mathbf{f}_{0}} & =\mathbf{f}_{0}\cdot(\mathbf{C}\mathbf{f}_{0})=\lambda^2.
\end{align}
\] Substitution into the passive and active strain-energy functions
gives \[
\begin{equation}
\Psi_{p}(I_{1},I_{4\mathbf{f}_{0}})  = \frac{a}{2b}\bigg(e^{ b(\lambda^2 +2\lambda^{-1}-3) }-1\bigg)  + \frac{a_{f}}{2b_{f}}\bigg(e^{ b_{f}(\lambda^2-1)^2_{+}}-1\bigg).
\end{equation}
\] The passive first Piola-Kirchoff stress components are \[
\begin{align}
(\mathbf{P}_{p})_{11} & =\frac{ \partial \Psi_{p} }{ \partial \lambda } +Jp(\mathbf{F}^{-T})_{11} \\
 & = \frac{a}{2}(2\lambda -2\lambda^{-2})e^{ b\left( \lambda^2 +2\lambda^{-1}-3 \right) }+ a_{f}(\lambda^2-1)_{+}2\lambda e^{ b_{f}(\lambda^2-1)^2_{+} }+\frac{Jp}{\lambda} \\
  & =a(\lambda-\lambda^{-2})e^{ b(\lambda^2+2\lambda^{-1}-3) }+2\lambda a_{f}(\lambda^2-1)_{+}e^{ b_{f}(\lambda^2-1)_{+}^2 }+p\lambda^{-1},\\
(\mathbf{P}_{p})_{33}=(\mathbf{P}_{p})_{22} & =\frac{ \partial \Psi_{p} }{ \partial \left( \frac{1}{\sqrt{ \lambda }} \right) }  + Jp(\mathbf{F}^{-T})_{22}\\
 & =\frac{ \partial \lambda }{ \partial \left( \frac{1}{\sqrt{ \lambda }} \right) } \frac{ \partial \Psi_{p} }{ \partial \lambda } +Jp(\mathbf{F}^{-T})_{22} \\
 & = (-2\lambda^{3/2})\left[ a(\lambda-\lambda^{-2})e^{ b(\lambda^2+2\lambda^{-1}-3) }+2\lambda a_{f}(\lambda^2-1)_{+}e^{ b_{f}(\lambda^2-1)_{+}^2 } \right] +p\sqrt{ \lambda },
\end{align}
\] where we have used the chain rule and \[
\begin{equation}
\frac{ \partial \lambda }{ \partial\left(  \frac{1}{\sqrt{ \lambda }} \right) } =\frac{ \partial \gamma^{-2} }{ \partial \gamma } =-2\gamma^{-3}=-2\lambda^{3/2}.
\end{equation}
\] The active Cauchy stress tensor is given by \[
\begin{equation}
\boldsymbol{\sigma}_{a}=\text{diag}\{T_{a}, \eta T_{a},\eta T_{a}\},
\end{equation}
\] and is also related to the first Piola-Kirchoff stress by \[
\begin{equation}
\mathbf{P}_{a}=J\boldsymbol{\sigma}_{a}\mathbf{F}^{-T}\implies \boldsymbol{\sigma}_{a}=\mathbf{P}_{a}\mathbf{F}^T.
\end{equation}
\]

We assume that the slab is unloaded, and we ignore body and inertia
forces, such that the active and passive stresses balances at all
points: \[
\begin{equation}
\mathbf{P}=\mathbf{P}_{p}+\mathbf{P}_{a}=0.
\end{equation}
\] Combination of \eqref{} gives \[
\begin{equation}
\text{diag}\{T_{a},\eta T_{a},\eta T_{a}\}=-\frac{1}{J}\mathbf{P}_{p}\mathbf{F}^T
\end{equation}
\] Taking the above equation for the first component gives \[
\begin{align}
T_{a} & =-\left[ a(\lambda-\lambda^{-2})e^{ b(\lambda^2+2\lambda^{-1}-3) }+2\lambda a_{f}(\lambda^2-1)_{+}e^{ b_{f}(\lambda^2-1)_{+}^2 }+p\lambda^{-1} \right] \lambda \\
 & =-a(\lambda^2-\lambda^{-1})e^{ b(\lambda^2+2\lambda^{-1}-3) }-2\lambda^2 a_{f}(\lambda^2-1)_{+}e^{ b_{f}(\lambda^2-1)_{+}^2 }-p,
\end{align}
\] and taking the second (and third) component gives \[
\begin{align}
\eta T_{a} & = - \left[ (-2\lambda^{3/2})\left[ a(\lambda-\lambda^{-2})e^{ b(\lambda^2+2\lambda^{-1}-3) }+2\lambda a_{f}(\lambda^2-1)_{+}e^{ b_{f}(\lambda^2-1)_{+}^2 } \right] +p\sqrt{ \lambda } \right] \frac{1}{\sqrt{ \lambda }} \\
 & =2a(\lambda^2-\lambda^{-1})e^{ b(\lambda^2+2\lambda^{-1}-3) }+4\lambda^2 a_{f}(\lambda^2-1)_{+}e^{ b_{f}(\lambda^2-1)_{+}^2 }-p.
\end{align}
\] We will further assume that the tension is only activated in the
fiber direction such that \(\eta=0\). Given an active tension \(T_{a}\),
we could calculate the corresponding values for \(\lambda\) and \(p\) by
finding the roots of the expressions \[
\begin{align}
L_{1}(\lambda,p;T_{a}) & =T_{a}+a(\lambda^2-\lambda^{-1})e^{ b(\lambda^2+2\lambda^{-1}-3) }+2\lambda^2 a_{f}(\lambda^2-1)_{+}e^{ b_{f}(\lambda^2-1)_{+}^2 }+p, \\
L_{2}(\lambda,p) & =2a(\lambda^2-\lambda^{-1})e^{ b(\lambda^2+2\lambda^{-1}-3) }+4\lambda^2 a_{f}(\lambda^2-1)_{+}e^{ b_{f}(\lambda^2-1)_{+}^2 }-p.
\end{align}
\] With these assumptions, we can describe the electrophysiological and
mechanical activity in the cuboidal slab with the equations \[
\begin{align}
%\label{eq:0dODE}
\frac{ \partial s }{ \partial t }  & =f(s,\lambda,\dot{\lambda},t), \\
%\label{eq:L1}
L_{1}(\lambda,p;T_{a}) & =0, \\
%\label{eq:L2}
L_{2}(\lambda,p)  & =0.
\end{align}
\]

\subsection{0D coupling without
feedback}\label{d-coupling-without-feedback}

One simple implementation is to assume that the right-hand sides of the
cell model ODEs are independent of \(\lambda\) and \(\dot{\lambda}\):
\(f(v,s,\lambda,\dot{\lambda},t)=f(v,s,t)\). Then, the values computed
from the cell model affect the mechanics equations, but the values
computed from mechanics equations do not affect the cell model. This can
be implemented by setting \(\lambda=1\) and
\(\frac{\text{d}\lambda}{\text{d}t}=0\) in the cell model for all \(t\).

The algorithm for solving this can be described as
0D\_weakcoupling(\(T\), \(\Delta t\)) 1. \(n=0\) 2. while \(t<T\) 1.
\(n+=1\) 2. \(t=n\cdot \Delta t\) 3. With an initial \(s_{n}\), compute
\(s_{n+1}\) using a forward generalized Rush-Larsen scheme on the system
\(\frac{ \partial s }{ \partial t }=f(s_{n},1,0,t)\). 4. From
\(s_{n+1}\), compute \(T_{a,n+1}\). 5. Find \(\lambda_{n+1}\) and
\(p_{n+1}\) by applying a root finding algorithm on
\eqref{eq:L1}-\eqref{eq:L2}. 6. Set
\(\dot{\lambda}_{n+1}=\frac{\lambda_{n+1}-\lambda_{n}}{\Delta t}\).

\begin{algorithm}
\caption{$0D\_weakcoupling(T, \Delta t)$}
\begin{algorithmic}[1]
    \State $n \gets 0$
    \While{$t < T$}
        \State $n \gets n + 1$
        \State $t \gets n \cdot \Delta t$
        \State With an initial $s_{n}$, compute $s_{n+1}$ using a forward generalized Rush-Larsen scheme on $\frac{ \partial s }{ \partial t }=f(s_{n},1,0,t)$.
        \State From $s_{n+1}$, compute $T_{a,n+1}$.
        \State Find $\lambda_{n+1}$ and $p_{n+1}$ by applying a root finding algorithm on \eqref{eq:L1}-\eqref{eq:L2}.
        \State $\dot{\lambda}_{n+1} \gets \frac{\lambda_{n+1}-\lambda_{n}}{\Delta t}$.
    \EndWhile
\end{algorithmic}
\end{algorithm}

\subsection{0D coupling with feedback}\label{d-coupling-with-feedback}

Now we include \(\lambda\) and \(\dot{\lambda}\) on the right-hand side
of the system of ODEs. Notice that \(\lambda\) relies on \(T_{a}\)
through the active strain-energy function \eqref{eq:activepsi}, and
\(T_{a}\) relies on \(\lambda\) through the active tension model
\eqref{eq:activetension}.

0D\_strongcoupling(\(T\), \(\Delta t\)) 1. \(n=0\) 2. while \(t<T\) 1.
\(n+=1\) 2. \(t=n\cdot \Delta t\) 3. With an initial \(s_{n}\), compute
\(s_{n+1}\) using a forward generalized Rush-Larsen scheme on
\(\frac{ \partial s }{ \partial t }=f(s_{n},\lambda_{n},\dot{\lambda}_{n},t)\).
4. From \(s_{n+1}\), compute \(T_{a,n+1}\). 5. Find \(\lambda_{n+1}\)
and \(p_{n+1}\) by applying a root finding algorithm on
\eqref{eq:L1}-\eqref{eq:L2}. 6. Set
\(\dot{\lambda}_{n+1}=\frac{\lambda_{n+1}-\lambda_{n}}{\Delta t}\).

\begin{algorithm}
\caption{$0D\_strongcoupling(T, \Delta t)$}
\begin{algorithmic}[1]
    \State $n \gets 0$
    \While{$t < T$}
        \State $n \gets n + 1$
        \State $t \gets n \cdot \Delta t$
        \State With an initial $s_{n}$, compute $s_{n+1}$ using a forward generalized Rush-Larsen scheme on $\frac{ \partial s }{ \partial t }=f(s_{n},\lambda_{n},\dot{\lambda}_{n},t)$.
        \State From $s_{n+1}$, compute $T_{a,n+1}$.
        \State Find $\lambda_{n+1}$ and $p_{n+1}$ by applying a root finding algorithm on \eqref{eq:L1}-\eqref{eq:L2}.
        \State $\dot{\lambda}_{n+1} \gets \frac{\lambda_{n+1}-\lambda_{n}}{\Delta t}$.
    \EndWhile
\end{algorithmic}
\end{algorithm}

We see that the solutions for \(\lambda,\dot{\lambda},T_{a},\) and \(p\)
starts oscillating rapidly. The reason for this oscillation can be
explained with the following example: Let's say that we find an increase
in \(T_{a}\) from the ODE system. This leads to a decrease of
\(\lambda\) through contraction. In Figure \ref{fig:land_hlambda}, we
see that \(h(\lambda)\) in \eqref{eq:activetension} has a positive
correlation to \(\lambda\). Then, in the next timestep, the decrease of
\(\lambda\) will lead to a decrease of \(T_{a}\).

The problem is that after \(T_{a,n+1}\) is computed from the ODE system,
it is held at that value for each iteration of root finding algorithm.
\#\# 0D monolithic coupling We can fix this issue by solving the ODE
system and updating \(T_{a}\) for every iteration of the root finding
algorithm.

0D\_monolithic(\(T\), \(\Delta t\)) 1. \(n=0\) 2. while \(t<T\) 1.
\(n+=1\) 2. \(t=n\cdot \Delta t\) 3. Find \(\lambda_{n+1}\) and
\(p_{n+1}\) by applying a root finding algorithm on
\eqref{eq:L1}-\eqref{eq:L2}, solving the ODE system
\(\frac{ \partial s }{ \partial t }=f(s_{n},\lambda_{n},\dot{\lambda}_{n},t)\)
in every iteration. 4. Set
\(\dot{\lambda}_{n+1}=\frac{\lambda_{n+1}-\lambda_{n}}{\Delta t}\).

\begin{algorithm}
\caption{$0D\_monolithic(T, \Delta t)$}
\begin{algorithmic}[1]
    \State $n \gets 0$
    \While{$t < T$}
        \State $n \gets n + 1$
        \State $t \gets n \cdot \Delta t$
        \State Find $\lambda_{n+1}$ and $p_{n+1}$ by applying a root finding algorithm on \eqref{eq:L1}-\eqref{eq:L2}, solving the ODE system $\frac{ \partial s }{ \partial t }=f(s_{n},\lambda_{n},\dot{\lambda}_{n},t)$ in every iteration.
        \State $\dot{\lambda}_{n+1} \gets \frac{\lambda_{n+1}-\lambda_{n}}{\Delta t}$.
    \EndWhile
\end{algorithmic}
\end{algorithm}
